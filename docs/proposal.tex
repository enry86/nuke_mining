\documentclass[12pt,a4paper]{article}
\usepackage[italian]{babel}
\usepackage[latin1]{inputenc}

\title{Data Mining project proposal:\\
Radioxenon monitoring for verification of the Comprehensive nuclear test ban
treaty}
\author{Gasparella Luca, Sartori Enrico}
\date{A.y. 2008/2009}

\begin{document}
\maketitle
\section{Problem Description}
The project we want to develop has been proposed as competition in the
IEEE International conference on Data Mining in 2008, its objective is 
to train a classifier able to find the evidence of a nuclear weapon test,
analyzing the concentration of some radioactive noble gas in the atmosphere.
The relative combination of these gas concentration can give a fingerprint
of the generating source (power plants, medical isotopes or different kind
of explosions).
The website containing the contest description is 
http://www.cs.uu.nl/groups/ADA/icdm08cup/index.html

The dataset is the catalog of four isotopes concentration levels collected 
from different stations. A training dataset, which includes the source of
the gas concentrations (Background or Background + Explosion), is provided
in order to train the classifier, along with a testing dataset, containing 
only the concentration levels.

\section{Solution Approach}
The solution will be about constructing a reliable classifier, able to 
distinguish the typical "fingerprint" of a nuclear explosion. This can be
achieved analyzing the frequency of certain values in the gas concentration
correlated to explosion events.

The winner of the 2008 contest provided a solution based on a technique called
random decision tree, which is described in the paper "Is random model better?
On its accuracy and efficiency" by Wei Fan et al. 2003 reachable at 
http://www1.cs.columbia.edu/~wfan/PAPERS/ICDM03rdt.pdf

This approach is about the construction of N different decision trees, through
the random picking of feature to be evaluated in each node.
Some slides describing the application of this technique to this problem are
here: http://www1.cs.columbia.edu/~wfan/PPT/ICDM08Contest.ppt

We are interested in providing our implementation of this approach, since the
code of their solution is not publicly available, and then compare our
performances with theirs.



\section{Project Plan}
The first phase of the project will be dedicated to analyze the data in 
order to retrieve from the dataset the necessary informations over the
gas concentration relative to an explosion.

These informations are needed in order to train the classifier, possibly
tuned for managing the differences between the data collected by the 
different stations.

Then is possible to develop a clustering strategy for finding the different
kind of explosions. 

We will anyway look for some other larger dataset, possibly some sampling 
over the last decades, in order to use and test our classifier with real 
historical data. 

Furthermore is possible to find different clusters which can group the 
datapoints by different kind of explosions.

\end{document}
