\documentclass{acm_proc_article-sp-sigmod07}

\begin{document}

\title{Framework for comparison between classification algorithms}
\numberofauthors{2}

\maketitle

\begin{abstract}
This paper presents a framework useful for the comparison between
classification algorithms.
The main evaluation measures used are the time used for building the
classification structure, the space consumed by the structure and the
accuracy of the classification provided.

The example provided here is the evaluation of the performances of a
generic decision tree and the Random Decision Tree algorithm by Wei Fan.
\end{abstract}

\section{INTRODUCTION}
The main objective of this work is to measure and compare the performances
of different classification algorithms. In particular we have focused on
the Random Decision Tree algorithm, comparing its performances with a
decision tree based on Information Gain.

During the development of the project we took care of two different
aspects, in fact we provided an implementation of the algorithms tested
and we have implemented a framework able to launch and measure the
performance of the different algorithms.

The evaluation of the algorithm is based on three measures: the time spent
for build the classification structure, the space needed to store and use
this structure and the accuracy of the algorithm.

The accuracy is defined as the percentage of correctly classified
datapoints in a dataset.
For allowing the analysis of the accuracy on every dataset, our framework
integrates the possibility to perform a cross validation over the dataset
using a user chosen percentage of datapoints for training ant the other
for testing.

The testing environment realized permits the setting of many parameters
used for tuning the classification task, in order to measure the influence
of those parameters on the overall performance.

The language chosen for the implementation of the platform and the
algorithms is Python. This choice has been made because of the relative
simplicity of the language which permitted us to focus more attention on
the algorithmic part rather than on solving more fine grain implementation
problems.

In the test section of this work we analyze how the different algorithms
behave when used on different dataset and with different parameters. 

\section{RELATED WORK}

The first algorithm used for comparison is a common Decision Tree, in fact
we wanted to have a comparison of the RDT approach with a commonly used
and known method.
The literature is rich of works about decision tree training and
implementation. In particular (ARTICLE INTRO) gives a complete background
of the basilar techniques used for the realization of such algorithms.

In this work particular attention is posed into the various measures used
to evaluate the best attribute to use to train the root of a decision
tree.
These techniques are presented in a more concrete way in a tutorial by
Andrew Moore (REFERENCE). In this presentation the algorithm used is
a decision tree based on information gain as splitting measure, and
presents some ways to optimize and prune the obtained tree.
For the information gain, the same author published a set of slides
in which presents this measure and the way in which it can be computed.

The second algorithm we choose for the comparison is the Random Decision
Tree presented by Wei Fan in (ARTICLE NUMBER)

\section{PROBLEM DEFINITION}
The main aim of this work is to produce a comparison between two different
approaches to the classification task. In particular we are focusing our
attention in analyzing the performances of a rather new approach (RDT),
which uses random choices ti build the decision tree, and a more common
algorithm based on information gain.

In order to to this we need to provide a reliable implementation of both
the algorithms, which permits us to perform some test and analysis on the
results acquired.

The measures used for comparison are three: 
\begin{itemize}
\item The time taken to build the tree and eventually perform the
classification three.
\item The space used to store the tree structure.
\item The accuracy obtained in the classification task.
\end{itemize}

The measure of the space used presented some difficulties, due to the
dynamic typing feature of the language used, in order to solve this we
implemented a method which measures the memory used by an object referring
to the correspondent primitive type of C. This method is explained in
details in following section.

The accuracy can be verified in case of cross validation, classifying a
set of previously known datapoints. This measure can vary depending on the
dataset characteristics. For this reason is important to compare the
results obtained by the two algorithms on the same datasets.

\section{PROPOSED APPROACH}
In this section we describe and discuss the choice taken during the
development of this project. We will focus first on the implementation of
the platform used for the algorithm comparison and the way it's intended
to be used.
Later the details about the implementation of the algorithms will be
clarified.

\subsection{Classification platform}




\end{document}
